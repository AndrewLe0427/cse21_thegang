
\documentclass[10pt,letterpaper,unboxed,cm]{article}
\usepackage[margin=1in]{geometry}
\usepackage{graphicx}
\usepackage{enumerate, comment}
\usepackage{adjustbox}
\usepackage{amsmath,mathabx}

\newcommand{\st}{~\mid~}
\newcommand{\ind}{$~~~~~$}




\begin{document}

\begin{center}
{\LARGE CSE 21: Math/Algorithms \& Systems Analysis}\\
{\Large Assignment 1}\\
{\large Due: Monday, October 6, 2025 at 11:59pm}
\end{center}


{\large Contributors:}
\begin{itemize}
    \item Jason Wang, jaw073@ucsd.edu
    \item Joshua Citrin, jcitrin@ucsd.edu
    \item Benjamin Michael, bemichael@ucsd.edu
\end{itemize}

\emph{(Note: for justifying counting arguments, a good rule of thumb is to explain how you came up with every term and factor of your answer. You can leave
your answer in terms of exponentials rather than compute the exact numerical value.)}

\begin{enumerate}
\item
\begin{enumerate}
\item (6 points)
Use regular induction to prove the following identity for all $n\geq 0$:

$$\sum_{k=0}^n k2^k = (n-1)2^{n+1} + 2$$

Proof:
\begin{enumerate}
    \item Claim: $\sum_{k=0}^{n} k2^k = (n-1)2^{n+1} + 2$ for all integers $n \geq 0$
    \item Base Case: $n=0$
    \begin{align*}
        \sum_{k=0}^{1} k2^k = 0*2^0 + 1*2^1 = 2\\
        (1 - 1)2^{1+1} + 2 = 0*2^2 + 2 = 2
    \end{align*}
    Both sides of the equation holds the equality when $n = 0$
    \item Induction Hypothesis:
    
    
    Assume the following expression for some integer $r \geq 0$
    $$
    \sum_{k = 0}^{r} k2^k = (r-1)2^{r+1} + 2
    $$
    \item Induction Step:

    
    Start from $\sum_{k=0}^{n+1}k2^k = (n+1-1)2^{n+1+1}+2$,
    \begin{align*}
        \sum_{k=0}^{r+1}k2^k = (r+1-1)2^{r+1+1}+2 \\
        (r+1)2^{r+1} + \sum_{k=0}^r k2^k = 2^{r+2} + 2(r-1)2^{r+1} + 2
    \end{align*}
    Since by the Induction Hypothesis we know that
    $$\sum_{k=0}^{n}k2^k = (n-1)2^{n+1}+2$$
    Then we can cancel the 2 terms on both sides and simplify as the following:
    \begin{align*}
        (r+1)2^{r+1} = (r-1)2^{r+1}+2^{r+2} \\
        r2^{r+1} + 2^{r+1} = r2^{r+1} - 2^{r+1} + 2^{r+2} \\
        2^{r+1} = -2^{r+1}+2*2^{r+1} \\
        2^{r+1} = (-1 + 2)2^{r+1} \\
        2^{r+1} = 2^{r+1}
    \end{align*}
    Which both sides of the equality are the same for any integer value $r \geq 0$
    \item Conclusion:

    
    Since our claim holds for $n=0$ and 
    $$\sum_{k = 0}^{r} k2^k = (r-1)2^{r+1} + 2 \xrightarrow{} \sum_{k=0}^{r+1}k2^k = (r+1-1)2^{r+1+1}+2$$
    for $n \geq 0$, by induction, 
    $$\sum_{k = 0}^{r} k2^k = (r-1)2^{r+1} + 2$$
    for all integers $n \geq 0$

    
\end{enumerate}

\item (6 points)
Use regular induction to prove the following identity for all $n\geq 2$:

$$\sum_{k=1}^n \left(1/\sqrt{k}\right) > \sqrt{n}$$

Proof:
\begin{enumerate}
\item Claim: 
$$
\sum_{k=1}^{n}(\frac{1}{\sqrt{k}}) > \sqrt{n}
$$
\item Base Case: $n = 2$
\begin{align*}
    \sum_{k=1}^{2}(\frac{1}{\sqrt{k}}) >\sqrt{2}\\
    \frac{1}{\sqrt{1}} + \frac{1}{\sqrt{2}} = \sqrt{2}
\end{align*}
Since the $\frac{1}{\sqrt{2}}$ is a positive number, it is valid to subtract $\frac{1}{\sqrt{2}}$ on both sides without switching the inequality.
\begin{align*}
    1 > \sqrt{2}-\frac{1}{\sqrt{2}}\\
    \sqrt{2} > 2 - 1\\
    \sqrt{2} > 1\\
    1.41... > 1
\end{align*}
The inequality holds when $n = 2$
\item Induction Hypothesis:

Assume the following expression for some integer $r \geq 2$
$$
    \sum_{k=1}^{r}(\frac{1}{\sqrt{k}}) > \sqrt{r}
$$
\item Induction Step:


Start from $\sum_{k=1}^{r+1}(\frac{1}{\sqrt{k}}) > \sqrt{r+1}$,
\begin{align*}
    \sum_{k=1}^{r+1}(\frac{1}{\sqrt{k}}) > \sqrt{r+1}\\
    \frac{1}{\sqrt{r+1}}+\sum_{k=1}^{r}(\frac{1}{\sqrt{k}}) > \sqrt{r+1}
\end{align*}
Since by the Induction Hypothesis we know that
$$
\sum_{k=1}^{r}(\frac{1}{\sqrt{k}}) > \sqrt{r}
$$
If we can prove $\frac{1}{\sqrt{r+1}}+\sqrt{r} > \sqrt{r+1}$ is valid, then this expression $\frac{1}{\sqrt{r+1}}+\sum_{k=1}^{r}(\frac{1}{\sqrt{k}}) > \sqrt{r+1}$ must also be true
Therefore we can replace the $\sum_{k=1}^{r}(\frac{1}{\sqrt{k}})$ with $\sqrt{r}$ and continue the proof.
\begin{align*}
    \frac{1}{\sqrt{r+1}} + \sqrt{r} > \sqrt{r+1}\\
    \frac{1}{\sqrt{r+1}} + \frac{\sqrt{r}\sqrt{r+1}}{\sqrt{r+1}} > \sqrt{r+1}\\
    \frac{1 + \sqrt{r}\sqrt{r+1}}{\sqrt{r+1}} > \sqrt{r+1}\\
\end{align*}
We can multiply both sides with $\sqrt{r+1}$ under the condition that $r \geq 2$ as it is always a positive number, therefore the inequality sign will not be flipped.
\begin{align*}
    1+\sqrt{r}\sqrt{r+1} > r+1\\
    \sqrt{r}\sqrt{r+1} > r\\
    \sqrt{r^2+r} > r\\
    r\sqrt{1+\frac{1}{r}} > r
\end{align*}
Under the condition of $r \geq 2$, r is always a non-zero positive integer, therefore we can divide both side of the inequality without flipping the inequality sign
\begin{align*}
    \sqrt{1+\frac{1}{r}} > 1
\end{align*}
Since $\sqrt{1 + \frac{1}{r}}$ and $1$ are both $\geq 1$, it is valid to square both sides of the inequality without effecting the inequality sign
\begin{align*}
    (\sqrt{1 + \frac{1}{r}})^2 > (1)^2\\
    1+\frac{1}{r} > 1
\end{align*}
Under the condition that $r \geq 2$, $\frac{1}{r} >0$, Which $1 + \text{x} > 1$ where $x$ is any number that is greater than 0
\item Conclusion:

Since our claim holds for $n = 2$ and 
$$
    \sum_{k=1}^{r}(\frac{1}{\sqrt{k}}) > \sqrt{r} \rightarrow{}\sum_{k=1}^{r+1}(\frac{1}{\sqrt{k}}) > \sqrt{r+1}
$$
for any $r \geq 2$
By induction,
$$
    \sum_{k=1}^{n}(\frac{1}{\sqrt{k}}) > \sqrt{n}
$$
for all $n \geq 2$.
\end{enumerate}
\end{enumerate}

\item
A password is a string over the alphabet of 72 characters consisting of the 26 uppercase letters, the 26 lowercase letters, the 10 digits, and the 10 special characters: $\{!,@,\#,\$,\%,\&,?,+,=,-\}$. (\emph{2 points for correct expression and 2 points for justification})

\begin{enumerate}


\item
(4 points)
How many 8-character passwords have at least 2 letters (they can each be uppercase or lowercase)?
$$
    72^8-20^8-(\binom{8}{1}52*20^7)
$$
Explanation:
\begin{itemize}
    \item $72^8$: represents all possible password combinations possible
    \item $20^8$: represents the possible password combinations with no letters (uppercase or lowercase)
    \item To represent the possible password combinations with only 1 letter (uppercase or lowercase):
    \begin{itemize}
        \item $\binom{8}{1}$: of the 8 positions of the password, which one will contain the letter
        \item $52$: of the 52 possible letters (uppercase and lowercase), which one will be the letter included in the password
        \item $20^7$: represents all the rest of the positions filled without any letters
    \end{itemize}
\end{itemize}
Since the possibilities are composed of
\begin{align*}
\underbrace{\text{total amount of possible passwords}}_{72^8} = \underbrace{\text{passwords without a letter}}_{20^8}\\ 
+ \\
\underbrace{\text{passwords with only 1 uppercase or lowercase letter}}_{\binom{8}{1}*52*20^7}\\ 
+ \\
\underbrace{\text{passwords with at least 2 uppercase or lowercase letters}}_{\text{Result}}
\end{align*}
we can find the possible 8-character passwords with at least 2 letters (uppercase or lowercase) by subtracting the total count by the number of possible combinations with $\leq 1$ letter in the password.

\item
(4 points)
How many 8-character passwords avoids the word COUNT all in uppercase?
$$
72^8-(4*72^3)
$$
Explanation:
\begin{itemize}
    \item $72^8$: represents all possible password combinations possible
    \item To represent the possible combinations with the word "COUNT" in all uppercase letters included within the password
    \begin{itemize}
        \item $4$: represents the 4 possible positions where the word "COUNT" can be placed
        \item $72^3$: represents the possible values that can be placed in the remaining 3 slots
    \end{itemize}
\end{itemize}
Since the possibilities are composed of 
\begin{align*}
    \underbrace{\text{total amount of possible passwords}}_{72^8} = \underbrace{\text{passwords with the combination "COUNT" within the password}}_{4*72^3}\\ 
    + \\
    \underbrace{\text{passwords without the word "COUNT"}}_{\text{result}}
\end{align*}
We can find the possible 8-character passwords without the word "COUNT" in all uppercase letters by subtracting the total count by the number of possible passwords including the word "COUNT" in all uppercase.

\item
(4 points)
How many 8-character passwords consist of 4 different characters with 2 copies of one, 2 copies of another, 3 copies of another and a single copy of the last?
$$
    \binom{72}{4}\binom{4}{1}\binom{8}{3}\binom{3}{2}\binom{5}{2}\binom{3}{2}\binom{1}{1}\binom{1}{1}
$$
Explanation:
\begin{itemize}
    \item $\binom{72}{4}$: choosing 4 characters out of the 72 possible characters
    \item $\binom{4}{1}$: choosing 1 character out of the 4 to be the character copied for 3 times
    \item $\binom{8}{3}$: choosing 3 slots out of the 8 slots to put the character that has been duplicated 3 times
    \item $\binom{3}{2}$: choosing 2 character out of the remaining 3 to be the character to be copied 2 times, the remaining character is by default the characters that is unique
    \item $\binom{5}{2}$: choosing 2 slots out of the 5 remaining to place one of the characters copied 2 times
    \item $\binom{3}{2}$: choosing 2 slots out of the 3 remaining to place the other character that has been copied 2 times
    \item The first $\binom{1}{1}$: choosing 1 character out of the remaining 1 to be the unique character
    \item The second $\binom{1}{1}$: choosing 1 slot out of the 1 remaining to place the last character
\end{itemize}


\emph{Ex: Ag3Ag3a3, 12341244, aaAAbbbB...}

\item
(4 points)
How many 8-character passwords have exactly 3 different special characters?
$$
    \binom{10}{3}\binom{8}{3}P(3, 3)*65^5
$$
\begin{enumerate}
    \item $\binom{10}{3}$: Choose the 3 special characters to be included in the password
    \item $\binom{8}{3}$: Choose the 3 slots out of 8 slots to place the special characters for the first time
    \item $P(3, 3)$: Ways to orient the 3 slots with the 3 special characters
    \item $65^5$: The remaining 5 slots have all the letters, numbers, and the 3 special characters to choose from
\end{enumerate}

\item
(4 points)
How many 8-character passwords consist of 8 different letters (they each can be uppercase or lowercase, but they must be different letters. For example, you cannot have AaBbCcDd but it is fine to have ZpxTaHwy.)?
$$
    \binom{26}{8}P(8, 8) * 2^8
$$
\begin{enumerate}
    \item $\binom{26}{8}$: Choose 8 letters (regardless of uppercase or lowercase) out of the 26 letters to include in the password
    \item $P(8, 8)$: Ordering 8 different letters(regardless of uppercase or lowercase) in 8 slots of the password 
    \item $2^8$: At each slot of the password, there are 2 options to the letter (uppercase or lowercase), and spans through 8 slots in the password
\end{enumerate}
\end{enumerate}

\item (4 points each) For each expression, describe a set of objects that is counted by the expression and include your reasoning.

For Example: Given the expression: $8*9^7$, here are a few sets that it could possibly count:

\begin{itemize}
\item
The number of strings of digits of length 8 with exactly one occurrence of 0.
\begin{quote}
(Reasoning:) The factor $8$ tells us which position the 0 is in and the factor $9^7$ tells us the rest of the 7 positions using the remaining 9 digits.
\end{quote}
\item
The number of strings of length 8 that start with a letter $\{A,B,C,D,E,F,G,H\}$ and end with 7 digits from 1 to 9.
\begin{quote}
(Reasoning:) The factor $8$ tells us which letter is in the first position and the factor $9^7$ tells us the rest of the 7 positions using the digits 1 to 9.
\end{quote}
\item
The number of strings of digits of length 8 that start with a digit $\{0,1,2,3,4,5,6,7\}$ does not repeat the same digit in two consecutive positions
\begin{quote}
(Reasoning:) The factor $8$ tells us which digit is in the first position and the factor $9^7$ tells us each of the next 7 positions using the remaining 9 digits that are different than the digit before.
\end{quote}
\end{itemize}

\begin{enumerate}
\item

$$26*8*10^7$$

The number of strings of length 8 that includes exactly 1 lowercase letter and the rest are all digits

Reasoning:
\begin{itemize}
    \item $26$: choose 1 letter from all 26 lowercase letters
    \item $8$: choose 1 out of 8 slots to place the lowercase letters
    \item $10^7$: choose the rest of the slots with digits 0-9
\end{itemize}
\item

$$10^3(26 + 10)^{5}{8\choose 3}$$

The number of strings of length 8 that includes at least 3 digits

Reasoning:
\begin{itemize}
    \item $10^3$: Choosing 3 digits out of 10 digits (repeat allowed)
    \item $\binom{8}{3}$: Choose where the 3 digits will go in the 8 slots available
    \item $(26+10)^5$: Choosing any lowercase letters or digits for the remaining 5 slots
\end{itemize}
\item

$$(26 + 10)^{8} - 26^8$$

The number of strings of length 8 that includes at least 1 digit

Reasoning:
\begin{itemize}
    \item $(26+10)^8$: The total amount of possible combinations to form a string of length 8 with a set of lowercase letters and digits 0-9
    \item $26^8$: The amount of possible combinations to from a string of length 8 only consisting of lowercase letters
\end{itemize}
$$
\underbrace{\text{total combination}}_{(26+10)^8} = \underbrace{\text{combination with only lowercase letters}}_{26^8} + \underbrace{\text{combinations with at least 1 digit}}_{result}
$$

\item
$$26^8 + 10^8 + 10^4*26^4$$

The number of strings of length 8 that consists of either only lowercase letters, only digits, or exactly 4 lowercase and exactly 4 digits.

Reasoning: 
\begin{itemize}
    \item $26^8$: number of strings consisting of only lowercase letters
    \item $10^8$: number of strings consisting of only digits 0-9
    \item $10^4 * 26^4$: number of strings consisting of only 4 digits 0-9 and only 4 lowercase letters
\end{itemize}
Since each term is added, it means each combination is one of the possibility that this string can have, hence the keyword "or".
\end{enumerate}

\item
(please include justification)
\begin{enumerate}
\item
( 4 points)

How many different ways are there to arrange the letters in the string UNITEDSTATES?

In the string UNITEDSTATES, there consists of
\begin{itemize}
    \item U 1
    \item N 1
    \item I 1
    \item T 3
    \item E 2
    \item D 1
    \item S 2
    \item A 1
\end{itemize}
with the total length of the string = 12, we can express the arrangement of these letters as the following:
$$
    \binom{12}{1}\binom{11}{1}\binom{10}{1}\binom{9}{3}\binom{6}{2}\binom{4}{1}\binom{3}{1}\binom{2}{2}
$$

\item
(4 points)

How many different ways are there to color the 8 \emph{vertices} of a cube with 8 different colors?

\begin{enumerate}
    \item Consider the first vertex, there are 8 colors to choose from
    \item Consider the second vertex, there are 7 colors left to choose from
    \item If we continue to go down in this pattern, there will be $P(8, 8)$ ways to color the 8 vertices of a cube with 8 different colors
    \item need to consider also the duplicates. 
    \begin{itemize}
        \item for each face of the cube that is oriented to face the top, there are 4 unique rotational symmetries
        \item There are a total of 6 faces that can be oriented to face the top
    \end{itemize}
\end{enumerate}
Therefore, the number of ways to formulate 8 vertices with 8 different colors is: 
$$
    \frac{P(8, 8)}{6 * 4}
$$
\item
(4 points)

How many different ways are there to make a necklace with 8 different colored beads (two necklaces are the same if you can manipulate one to look like the other.)

Imagine a necklace that forms a circle
\begin{enumerate}
    \item The first bead can be 1 of 8 colored bead to add in the necklace
    \item The second bead can be 1 or 7 colored bead remaining to add in the necklace
    \item if we continue to go down in this pattern, there will be $P(8, 8)$ ways to pick the colored beads
    \item need to consider to over counting
    \begin{itemize}
        \item When considering the top bead, due to the circle formation, we can rotate the beads 1 spot clockwise and that is considered a repeat, which has a total of 8 spots clockwise before returning to the original combination
        \item When considering the vertical line aligning the top and bottom beads, there can have 2 copies of the same necklace when flipped around along that line, which has 2 copies
    \end{itemize}
\end{enumerate}
Therefore, the number of ways to formulate 8 colored beads in a necklace is:
$$
    \frac{P(8, 8)}{8 * 2}
$$
\end{enumerate}
\end{enumerate}
\end{document}
